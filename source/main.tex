\documentclass[11pt,letterpaper]{article}

% One place for your macros and styling.
\usepackage{resume}

\begin{document}

{\LARGE \textbf{Your Name}}\\
\href{mailto:you@example.com}{you@example.com} \;|\;
\href{https://your-site.com}{your-site.com} \;|\;
City, ST \;|\; \href{https://github.com/yourhandle}{github.com/yourhandle}
\vspace{6pt}


\section{Highlights}
\begin{tightitemize}
  \item Open-source: authored tools with up to 688 GitHub stars (Imager).
  \item High-volume contributor; 2019 activity: 1,323 GitHub contributions (top percentile ranking).
  \item Built an experimental Markdown renderer on Apple TextKit2 (Swift).
\end{tightitemize}

\section{Experience}
\entry{Company or Client}{City, ST}{Role Title}{2024 -- Present}
\begin{itemize}\tightitem
  \item One-line impact statement with an outcome (performance, reliability, time saved, revenue, etc.).
  \item What you owned end-to-end (design \textrightarrow{} implementation \textrightarrow{} shipping \textrightarrow{} maintenance).
  \item Tools/stack in context (Rust, Swift, SQL, web scraping, etc.).
\end{itemize}

\entry{Another Company}{City, ST}{Role Title}{2022 -- 2024}
\begin{itemize}\tightitem
  \item Specific responsibility + result.
  \item Specific responsibility + result.
\end{itemize}


% IGNORE THIS
% \section{Projects}
% \project{Project Name}{https://github.com/you/project}{
One sentence: what it is and who it’s for.
\begin{itemize}\tightitem
  \item What you built that is non-trivial (parser, pipeline, crawler, etc.).
  \item Constraint you solved (scale, correctness, determinism, performance).
  \item Evidence (users, volume processed, benchmarks, shipped artifact).
\end{itemize}
}

\project{Another Project}{https://your-site.com/project}{
One sentence: what it is and why it matters.
\begin{itemize}\tightitem
  \item Measurable or concrete technical detail.
\end{itemize}
}


% IGNORE THIS
% \section{Education}
% % Omit this section entirely if you have nothing strong to say.
% If you keep it, keep it short and factual.

% Example 1: degree
% \entry{University Name}{City, ST}{B.S. in Computer Science}{2018 -- 2022}
% \begin{itemize}\tightitem
%   \item Relevant coursework: Compilers, Operating Systems, Databases
% \end{itemize}

% Example 2: no degree, but real training
% \entry{Independent Study / Training}{Remote}{Systems programming, compilers, data tooling}{2019 -- Present}
% \begin{itemize}\tightitem
%   \item Built production-grade tooling in Rust and Swift; focus on parsers, pipelines, and automation.
% \end{itemize}


\section{Skills}
\textbf{Languages:} Rust, Swift, TypeScript, SQL\\
\textbf{Systems:} CLI tooling, parsers/AST transforms, build pipelines, web scraping\\
\textbf{Data:} Postgres, schema design, ETL/ingestion, reporting\\
\textbf{Web:} HTML/CSS, static site generation, performance/SEO fundamentals\\
\textbf{Tools:} Git, Linux/macOS, Docker (basic), CI (basic)


\section{Postlude}
\begin{figure}[htbp]
  \centering
  \begin{tikzpicture}
    \node[inner sep=0pt, rounded corners=8pt, clip] (img)
      {\includegraphics[width=\linewidth]{../assets/SubscriptScreenshot.png}};
    \draw[rounded corners=8pt] (img.south west) rectangle (img.north east);
  \end{tikzpicture}

  \caption{The Subscript Note Taking App}
  \label{fig:photo}

  \captionsetup{justification=centering,font=small}
  \caption*{%
    \textit{Web publishing example:}\par
    \printlink{https://colbyn.github.io/old-school-chem-notes/dev/chemistry-1010---fall-2021/week-14-acids-and-bases.html}\par
    {\footnotesize (Shown: older build; no light/dark mode and missing later refinements.)}%
  }
  \caption{\textsc{Pre-LLM app}; designed, implemented, and iterated the hard way.}
\end{figure}

\end{document}
